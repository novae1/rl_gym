\documentclass[12pt,a4paper]{article}
\usepackage[utf8]{inputenc}
\usepackage[T1]{fontenc}
\usepackage[brazil]{babel}
\usepackage{geometry}
\usepackage{enumitem}
\usepackage{booktabs}
\usepackage{amsmath}
\usepackage{hyperref}
\geometry{margin=2.5cm}

\usepackage{amsfonts}

\title{\textbf{Trabalho: Análise Computacional de Jogos Estocásticos Cooperativos e Não Cooperativos}}
\author{Disciplina: Aprendizado por Reforço e  Controle  \\ Profs. Daniel Sadoc Menasché, Heudson Mirandola e Amit Bhaya}
\date{}

\begin{document}
\maketitle

\section*{Contexto}

Este trabalho visa aprofundar o entendimento de \textbf{equilíbrios em jogos estocásticos} aplicados a sistemas de comunicação com múltiplos agentes — por exemplo, enlaces sem fio sujeitos a interferência, controle de potência e políticas de admissão de pacotes. 

O notebook base (\texttt{team\_game\_vs\_non\_cooperative\_simple\_stochastic\_game.py}) apresenta três modos principais de interação estratégica:

\begin{enumerate}[label=\alph*)]
    \item \textbf{Jogo de Soma Zero (Zero-Sum)}
    \item \textbf{Jogo Não Cooperativo (Equilíbrio de Nash Descentralizado)}
    \item \textbf{Jogo de Time (Team Problem)}
\end{enumerate}

Cada modo é resolvido por meio de \textbf{programação linear iterativa}, e os resultados são comparados em termos de throughput físico agregado, políticas de transmissão e políticas de admissão.

\section*{Objetivos}

\begin{itemize}
    \item Compreender a formulação de jogos estocásticos envolvendo múltiplos agentes e restrições de potência e backlog.
    \item Explorar métodos alternativos de solução (Programação Linear versus \textbf{Problema de Complementaridade Linear – LCP}, resolvido via \textbf{algoritmo de Lemke}).
    \item Analisar efeitos paramétricos sobre o desempenho, estabilidade do buffer e comportamento do controlador de admissão.
    \item Identificar condições sob as quais o sistema opta por rejeitar pacotes, equilibrando throughput e estabilidade.
\end{itemize}

\section*{Atividades Propostas}

\subsection*{Parte 1 – Revisão e entendimento do notebook}

\begin{enumerate}[label=\arabic*.]
    \item Execute o notebook fornecido e explique o papel das funções e células principais:
    \begin{itemize}
        \item Definição do modelo e dos parâmetros.
        \item Solução dos três modos de jogo (Zero-Sum, Non-Coop, Coop).
        \item Cálculo e visualização das políticas ótimas.
        \item Avaliação do throughput físico agregado.
    \end{itemize}
    \item Descreva qualitativamente o significado das matrizes $c_1$ e $c_2$ e como elas mudam entre os modos.
\end{enumerate}

\subsection*{Parte 2 – Formulação e solução via algoritmo de Lemke (LCP)}

\begin{enumerate}[label=\arabic*.]
    \item O equilíbrio de Nash em jogos bilineares pode ser expresso como um \textbf{Problema de Complementaridade Linear (LCP)}:
    \[
        \text{Encontrar } x \ge 0 \text{ tal que } Mx + q \ge 0, \quad x^\top(Mx + q) = 0
    \]
    onde $M$ é a matriz de interação e $q$ contém os termos constantes das restrições.
    \item Derive $M$ e $q$ para a versão não cooperativa do jogo, considerando as distribuições ocupacionais ($\rho_1, \rho_2$).
    \item Resolva o LCP utilizando o \textbf{algoritmo de Lemke} (por exemplo, via biblioteca \texttt{lemkelcp} ou implementação própria).
    \item Compare a solução obtida com a solução iterativa baseada em programação linear, mostrando numericamente que ambas produzem os mesmos equilíbrios.
    \item Documente as diferenças em tempo de execução, número de iterações e precisão.
\end{enumerate}

\subsection*{Parte 3 – Inclusão de restrição sobre o backlog médio}

\begin{enumerate}[label=\arabic*.]
    \item Acrescente uma restrição de \textbf{tamanho médio do buffer}, impondo:
    \[
        \mathbb{E}[\text{Backlog}] \le B_{\text{max}}
    \]
    onde $B_{\text{max}}$ é um valor limite escolhido (por exemplo, 3 ou 4 pacotes).
    \item Mostre como essa restrição afeta o problema de otimização, podendo ser modelada como:
    \[
        \sum_{s,b,a,d} b \cdot \rho(s,b,a,d) \le B_{\text{max}}
    \]
    \item Reexecute as simulações para diferentes valores de $B_{\text{max}}$ e observe:
    \begin{itemize}
        \item Como as políticas de admissão mudam (em quais estados o sistema passa a rejeitar pacotes);
        \item Como o throughput e o atraso médio se alteram.
    \end{itemize}
    \item Inclua gráficos mostrando o \textbf{trade-off entre throughput e backlog}.
    \item Destaque cenários nos quais o controlador decide \textbf{não admitir pacotes}.
\end{enumerate}

\subsection*{Parte 4 – Análise paramétrica}

\begin{enumerate}[label=\arabic*.]
    \item Varie os parâmetros principais do sistema:
    \begin{itemize}
        \item Probabilidade de chegada (\texttt{ArrProb});
        \item Limites de potência ($v_1$, $v_2$);
        \item Número de estados de canal (\texttt{NLinkStates});
        \item Tamanho do buffer (\texttt{NBufferStates}).
    \end{itemize}
    \item Para cada variação:
    \begin{itemize}
        \item Gere e plote as políticas ótimas de potência e admissão;
        \item Recalcule o throughput físico e o backlog médio;
        \item Compare com os modos cooperativo e não cooperativo.
    \end{itemize}
    \item Discuta:
    \begin{itemize}
        \item Em que regimes o jogo cooperativo produz maior eficiência;
        \item Quando a restrição de backlog força o sistema a rejeitar admissões;
        \item O impacto da interferência e do custo de potência na estabilidade global.
    \end{itemize}
\end{enumerate}

\subsection*{Parte 5 – Relatório Final}

Monte um relatório contendo:
\begin{itemize}
    \item Breve introdução teórica ao problema;
    \item Descrição da metodologia (LP, LCP/Lemke e restrição de backlog);
    \item Gráficos comparativos de políticas e desempenho;
    \item Discussão dos resultados e conclusões.
\end{itemize}

\section*{Entrega}

\begin{itemize}
    \item Notebook executável (\texttt{.ipynb}) com comentários e gráficos;
    \item Relatório em PDF (máximo 6 páginas) resumindo os resultados.
\end{itemize}

 

\section*{Critérios de Avaliação}

\begin{center}
\begin{tabular}{@{}l c@{}}
\toprule
\textbf{Critério} & \textbf{Peso} \\
\midrule
Clareza e organização do código & 15\% \\
Utilização correta do LCP/Lemke & 20\% \\
Implementação da restrição de backlog médio & 20\% \\
Análise paramétrica e visualizações & 20\% \\
Discussão crítica e relatório final & 25\% \\
\bottomrule
\end{tabular}
\end{center}

\section*{Extensões sugeridas -- bonus extra}

\begin{itemize}
    \item Investigar políticas com custo ponderado de atraso (penalizando backlog elevado);
    \item Explorar versões assimétricas (diferentes limites de potência ou probabilidades de chegada);
    \item Estender o modelo para 3 jogadores (múltiplos enlaces interferentes).
\end{itemize}

\end{document}
